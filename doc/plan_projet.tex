\documentclass[11pt]{article}

\usepackage[utf8]{inputenc}
\usepackage[french]{babel}
\usepackage{amsmath}
\usepackage{amsfonts}
\usepackage{amssymb}
\usepackage{graphicx}
\usepackage{parskip}
\usepackage{multicol}

\begin{document}


\title{IFT3065 - Proposition de projet}
\date{Janvier 2012}
\author{Vincent Foley-Bourgon (FOLV08078309) \\
Eric Thivierge (THIE09016601)}
\maketitle

\abstract

Nous proposons d'écrire un compilateur Scheme en Scheme qui générera
du code pour une machine x86.  Ce compilateur sera une exploration
``en largeur'' du domaine de la compilation; plutôt que d'implanter
quelques fonctionnalités et de les développer en profondeur, nous
allons tenter d'implémenter plus de fonctionnalités, même si cela
implique de diminuer le degré de sophistication de notre compilateur.
Les deux membres de l'équipe pensent que cela aidera à obtenir une
bonne vision d'ensemble du fonctionnement d'un compilateur.

\section{Choix des langages}

\subsection{Langage source (Scheme)}

Nous avons décidé d'utiliser Scheme comme langage source.  La raison
principale est que Scheme est un des langages de haut niveau les plus
simple qui soit, ce qui permet d'implanter une grande partie du
langage dans un court lapse de temps.  De plus, il possède des
fonctionnalités que d'autres langages, comme Pascal, ne possèdent
pas. Par exemple:

\begin{itemize}
\item Fermetures
\item Récursivité terminale
\item Continuations
\item Macros
\end{itemize}


\subsection{Langage hôte (Scheme)}

Nous voulions un langage fonctionnel de haut niveau pour notre
compilateur.  Le langage OCaml était une option intéressante, étant
donné qu'il est similaire au langage SML utilisé dans le livre
``Modern Compiler Implementation in ML'' par Appel.  Son système de
typage statique aurait certainement été fort utile tout au long du
développement.

Comme un membre de l'équipe ne connaît pas OCaml, nous avons décidé
d'utiliser Scheme comme langage hôte.  Scheme est également un langage
de fonctionnel de haut niveau.  De plus, comme il s'agit du même
langage que le langage source, nous avons la possibilité d'écrire un
compilateur autogène.


\subsection{Langage cible (assembleur x86)}

Nous avons décidé de cibler le langage x86 pour notre compilateur.
Malgré que les deux membres de l'équipe ne soient pas familiers avec
cet assembleur, la popularité des machines x86 font de ce langage un
choix pratique.  Cela nous évite également de passer par un émulateur
pour un autre assembleur.


\section{Choix des fonctionnalités}

Afin de garder l'implantation du langage simple, nous avons éliminé
quelques fonctionnalités de Scheme pour notre projet.  Les
fonctionnalités suivantes ne se trouveront pas dans notre langage:


\begin{itemize}
\item Les nombres à virgule flottante
\item Les entiers à précision arbitraire
\item Unicode
\item L'instruction \emph{do}
\item \texttt{define-syntax}
\end{itemize}

Les fonctionnalités suivantes seront dans notre langage:


\begin{itemize}
\item \emph{lambda} et les fermetures
\item \emph{cond}
\item \emph{let}
\item Les paires
\item Les listes
\item La récursivité (optimisation des appels terminaux?)
\item Les commentaires commençant par un point-virgule
\item Les entiers (fixnum)
\item Les chaînes de caractères
\item Les caractères
\item Les symboles
\item Les booléens
\item Le mot clé \emph{begin}
\item Le mot clé \emph{set!}
\end{itemize}

Nous ne sommes pas certains si nous pourront inclure les
fonctionnalités suivantes:

\begin{itemize}
\item Les continuations
\item Les macros
\end{itemize}





\section{Choix des tâches}

\subsection{Grammaire, parseur et messages d'erreurs (niv-2)}

Comme nous implémentons notre propre compilateur, nous avons besoin
d'un analyseur lexical et syntaxique.  Nous avons décidé de les
implanter manuellement plutôt que d'utililiser des outils comme lex ou
bison, car nous aimerons connaître les détails de plus bas niveau de
ces outils.

Notre analyseur lexical aura les propriétés suivantes:


\begin{itemize}
\item Support pour ASCII seulement: il serait trop long et difficile
  de supporter correctement l'ensemble de Unicode, et nous nous
  limiterons donc à l'ensemble ASCII.
\item Le ``type'' de notre analyseur lexical sera \texttt{[char] ->
    [symbol]}; on prendra en paramètre une liste de caractères et on
  retournera une liste de symboles.  L'analyse lexicale ne sera pas
  concurrente avec l'analyse syntaxique.
\item Un symbole sera un triplet de la forme: \texttt{(type+valeur,
    ligne, colonne)}.  La ligne et la colonne seront utiliser pour
  donner des messages d'erreurs plus détaillés.
\end{itemize}

Notre analyseur syntaxique aura les propriétés suivantes:

\begin{itemize}
\item Ce sera un parseur LL(1) écrit par un algorithme de descente
  récursive, car c'est sans doute l'algorithme le plus simple à
  implanter manuellement.
\item Le parseur prendra en paramètre la liste de symboles générée par
  l'analyseur lexical et retournera un arbre de syntaxe abstraite.
\end{itemize}



\subsection{Fermetures (niv-2)}

Les fermetures sont l'une des fonctionnalités les plus importantes du
langage Scheme, et nous allons donc les implanter.

\subsection{Récupérateur de mémoire automatique (niv-2)}

Tout comme les fermetures, la gestion automatique de la mémoire est
une fonctionnalité importante de Scheme.  Nous allons donc implanter
un récupérateur simple.

\subsection{Types (niv-1)}

Scheme est un langage typé dynamiquement, donc les variables ne
possèdent pas de types, mais il est fortement typé, ce qui veut dire
qu'une expression tel que \texttt{(+ 3 'foo)} est sémantiquement
incorrecte.  Nous traiterons ces erreurs de type.

\subsection{Vivacité des variables (niv-1)}

\subsection{Propagation de constantes (niv-2)}

\subsection{Génération simple avec optimisation ``peephole'' (niv-2)}


\end{document}
