\documentclass[11pt]{report}

\usepackage{times}
\usepackage[utf8]{inputenc}
\usepackage[french]{babel}
\usepackage{amsmath}
\usepackage{amsfonts}
\usepackage{amssymb}
\usepackage{graphicx}
\usepackage{parskip}
\usepackage{multicol}
\usepackage{url}

\begin{document}

\newcommand{\sins}{\emph{SINS}}

\title{IFT3065 - Rapport sur SINS}
\author{Vincent Foley-Bourgon (FOLV08078309) \\
Eric Thivierge (THIE09016601)}
\maketitle

\abstract

\sins (Scheme IN Scheme) est un compilateur Scheme écrit dans le
langage Scheme dans le cadre du cours IFT3065 de la session hiver 2012
à l'Université de Montréal.  Ce rapport décrit la philosophie
directrice du projet, les différentes parties du compilateur, les
problèmes et difficultés rencontrés durant son écriture et les
solutions mises en place pour résoudre ces problèmes.

\chapter{SINS}

\section{Qu'est-ce que SINS?}

\sins (Scheme IN Scheme) est un petit compilateur Scheme écrit
dans le cadre d'un cours de compilation de 15 semaines.  Comme son nom
l'indique, il s'agit d'un compilateur pour Scheme écrit en Scheme.
\sins génère du code assembleur x86.

\sins est un véhicule permettant à ses auteurs de faire un
survol des différents aspects de la compilation.  La philosophie de
base est qu'étant donné le choix d'implanter une fonctionnalité en
profondeur ou d'en implanter deux avec des techniques moins
sophistiquées, on choisira la deuxième option à tout coup.  Nous
sommes d'avis que cette approche nous donnera une vision plus globale
d'un compilateur.  Les détails d'implantation plus avancés pourront
être appris, implantés et maîtrisés dans le cadre de stages, de
projets personnels, de recherches et d'emplois.


\section{Design de SINS}

\sins suit un modèle en pipeline où chaque étape de la compilation a
son propre module et où chaque module a comme entrée la sortie du
module précédent.  Chaque module effectue (si possible) une seule
transformation afin de garder la complexité du module la plus basse
possible.  Cela implique que \sins fait plus de passes sur le code et
qu'il est plus lent à l'exécution qu'un compilateur qui fusionne
plusieurs phases en une.  Dans le cadre d'un projet éducationnel,
nous jugeons que cette contrainte est acceptable.


\section{Milestone du 13 février 2012}

En date du 13 février 2012, les fonctionnalités suivantes sont
implantées:

\begin{itemize}
\item L'analyse lexicale
\item Le lecteur
\item La vérification syntaxique des formes spéciales
\end{itemize}

Notre code se trouve sur le site GitHub, et le code en date du 13
février 2012 peut être visionné à l'adresse suivante:
\url{http://www.github.com}


\chapter{Analyse lexicale}

La première phase habituelle d'un compilateur est de transformer un
flux de caractères représentant un programme source en un flux de
lexèmes.  Dans ce flux sortant, les espaces blancs et les commentaires
sont éliminés, et les caractères représentant des nombres ou des mots
sont regroupés ensemble pour une analyse syntaxique plus facile.

Il existe de nombreux outils automatisés qui permettent de faire
l'analyse lexicale; \emph{lex}, \emph{flex} et \emph{silex} sont de
tels outils.  Malgré leur disponibilité, nous avons décidé de ne pas
utiliser ces outils et de plutôt créer notre propre analyseur lexical.
Cette décision a été prise afin de mieux comprendre les détails de
plus bas niveau, ainsi que les difficultés d'implantation de cette
phase.

\section{Structures de données}

Afin de faire l'analyse lexicale d'un programme source, deux
structures de données principales ont été nécessaires: un type flux
(\emph{stream}), et un type lexème (\emph{token}).

\subsection{Stream}

La fonction \emph{make-stream} prend en entrée une chaîne de
caractères et retourne un \emph{stream}.  La fonction
\emph{make-stream-from-port} prend en entrée un port d'entrée et
retourne un \emph{stream}.  Un \emph{stream} est une structure de
données possédant les 4 opérations suivantes:

\begin{itemize}
\item \emph{next}: cette fonction retourne le caractère courant du
  flux.  Si le flux est terminé, elle retourne le caractère nul.
\item \emph{advance}: cette fonction fait avancer le curseur d'un
  caractère dans le flux et met à jour les attributs \emph{line} et
  \emph{col}.  \emph{Advance} n'a pas de valeur de retour.
\item \emph{line}: retourne la ligne courante.
\item \emph{col}: retourne la colonne courante.
\end{itemize}


\subsection{Token}

Un lexème est représenté par une liste ayant la forme suivante:

\begin{verbatim}
('token <(type . valeur)> <ligne> <colonne>)
\end{verbatim}

Le premier élément est une étiquette qui permet de distinguer un
lexème d'un autre type de données.  La paire en deuxième position
donne l'information sur le type du lexème (ex.: mot-clé,
identificateur, etc.) et la valeur qui lui est associée.  Les
troisième et quatrième champs conservent la position du début du
lexème.

Une erreur dans un lexème (par exemple: constante de caractère
invalide) est représentée par une liste ayant la forme suivante:

\begin{verbatim}
('token-error <ligne> <colonne>)
\end{verbatim}


On peut manipuler les lexèmes à l'aide des fonctions suivantes:

\begin{itemize}
\item \emph{make-token}: cette fonction permet de créer un lexème en
  donnant son type et sa valeur, sa ligne, et sa colonne.  Si
  type/valeur est \#f, \emph{make-token} retourne un erreur de lexème.
\item \emph{token?}: un prédicat qui détermine si un objet représente
  un lexème.
\item \emph{token-error?}: un prédicat qui détermine si un objet
  représente une erreur dans un lexème.
\item \emph{token-type}: retourne le type du lexème.
\item \emph{token-value}: retourne la valeur du lexème.
\item \emph{token-symbol}: retourne le type et la valeur d'un lexème.
\item \emph{token-line}: retourne la ligne où commence ce lexème.
\item \emph{token-col}: retourne la colonne où commence ce lexème.
\end{itemize}

\section{Algorithme d'analyse}

La fonction \emph{get-token} prend en entrée un flux de caractères et
retourne le prochain lexème.  Voici une explication de son
fonctionnement.

Tout d'abord, \emph{get-token} commence par ignorer les espaces blancs
et les commentaires.  Une fois que le flux se trouve sur un caractère
non-blanc, on effectue une analyse de cas.  Si le caractère courant
est un des terminaux suivants du langage, on retourne un lexème
représentant ce symbole: \texttt{(}, \texttt{)}, \texttt{,},
\texttt{,@}, \texttt{'}, \texttt{`}, et le caractère nul.

Dans le cas où le caractère est une guillemet, on accumule les
caractères du flux jusqu'à ce qu'on consomme la guillemet fermante (en
faisant attention aux séquences d'échappement).

Dans le cas où le caractère est un dièse, on essaye de lire, en ordre,
la constante vraie (\#t), la constante fausse (\#f) ou un caractère
(\#\textbackslash).  Dans le cas du caractère, si le caractère suivant
le backslash n'est pas alphabétique, on retourne immédiatement le code
ASCII de ce caractère.  Si le caractère est alphabétique, on tente de
lire un nom de caractère.  Nous supportons les noms ``nul'', ``tab'',
``space'' et ``newline''.  Si le nom possède un seul caractère, on
retourne le code ASCII associé à ce caractère.  Dans tous les autres
cas, on retourne \#f pour signaler une erreur.

Si le caractère est un chiffre, on lit un nombre.  Cela empêche des
identificateurs débutant par un chiffre comme \emph{2pi} par exemple.
Cette limitation est acceptable dans notre compilateur (et est
conforme avec la syntaxe décrite dans la section 7.1.1 du standard
R5RS).

Dans le cas où le caractère est un caractère qui fait partie de
l'alphabet des identificateurs, on procède en deux étapes:

\begin{enumerate}
\item On lit tous les caractères identificateurs;
\item On classifie le symbole et construisons le lexème approprié.
\end{enumerate}

Si le symbole représente un mot clé, on retourne un lexème de type
\emph{keyword}; autrement, on retourne un lexème de type
identificateur.

Finalement, pour tous les autres caractères, on retourne une erreur de
lexème.

La fonction \emph{lex} permet de prendre en entrée un flux de
caractères et de retourner une liste de lexèmes.  Les fonctions
\emph{lex-from-string} et \emph{lex-from-file} créent un flux à partir
d'une chaîne de caractères et d'un fichier source respectivement et
font ensuite appel à \emph{lex}.


\chapter{Analyse syntaxique}

Tout comme l'analyse lexicale, il existe de nombreux outils pour faire
l'analyse syntaxique d'un flux de lexèmes; Yacc, Bison, LALR-SCM et
ANTLR sont de tels outils.  Et tout comme pour la phase d'analyse
lexicale, afin de mieux comprendre les détails d'implantation de cette
phase et les difficultés qui y sont associées, nous avons décidé de ne
pas utiliser un de ces outils et de plutôt créer notre propre module
d'analyse syntaxique.

Comme la syntaxe de Scheme est simple et peut être exprimée par une
grammaire LL(1), il est naturel de créer cette phase par descente
récursive.  De plus, comme la récursion est omniprésente en Scheme et
que le langage possède des symboles, Scheme est un très bon choix pour
faire cette phase.

Les sections suivantes décriront les différentes phases de l'analyse
lexicale, les structures de donnés utilisées ainsi que certains
détails d'implantation que nous jugeons intéressants.

\section{Le lecteur}

Une caractéristique propre aux langages Lisp est le lecteur
(\emph{reader} en anglais); le lecteur prend en entrée une chaîne de
caractères représentant une structure Lisp et retourne la
représentation arborescente de cette structure.

Dans cette phase, les symboles sont auto-évalués (et non pas traités
comme des variables) et les listes sont vues non pas comme des appels
de fonctions ou des formes spéciales, mais simplement comme des listes
de symboles.  Voici un exemple d'interaction avec la fonction
\emph{read} dans l'interpréteur de Gambit:

\begin{verbatim}
> (read (open-input-string "allo"))
allo
> (read (open-input-string "(lambda)"))
(lambda)
> (read (open-input-string "(sins est un (bon) compilateur)"))
(sins est un (bon) compilateur)
\end{verbatim}

La seule vérification syntaxique faite par le lecteur est de s'assurer
que le parenthèsage des expressions est correct.  Aucune vérification
syntaxique plus approfondie n'est faite (ex.: qu'une forme lambda soit
suivie d'une liste de paramètres puis d'expressions à exécuter).  Ceci
est nécessaire pour la prochaine phase de l'analyse syntaxique,
l'expansion des macros.


\section{L'expansion de macros}

Cette phase n'est pas encore implantée.


\section{La vérification syntaxique}

Dans cette phase, \sins vérifie que les différentes formes spéciales
ont la bonne forme syntaxique.  Elle s'appuie sur la grammaire définie
dans la section 7.1.3 du standard R5RS de Scheme.

\end{document}
